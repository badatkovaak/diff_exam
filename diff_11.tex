\documentclass{article}
\usepackage[T2A, T1]{fontenc}
\usepackage{fancyhdr}
\usepackage{setspace}
\usepackage{titlesec}
\usepackage{amsmath}
\usepackage{amssymb}
\usepackage{amsthm}
\usepackage[english, russian]{babel}

\setlength{\parindent}{0pt}
\setlength{\parskip}{0pt}

\theoremstyle{plain} \newtheorem*{theorem*}{Теорема}
\theoremstyle{plain} \newtheorem{theorem}{Теорема}[section]
% \theoremstyle{remark} \newtheorem{definition}{Definition}[section]
% \theoremstyle{remark} \newtheorem{notation}{Notation}[section]
\theoremstyle{defintition} \newtheorem*{corollary*}{Следствие}
% \theoremstyle{definition} \newtheorem{corollary}{Следствие}[section]
\theoremstyle{remark} \newtheorem*{example*}{Пример}
\theoremstyle{remark} \newtheorem*{remark*}{Замечание}


\begin{document}

\abovedisplayskip=0pt
\belowdisplayskip=9pt
\abovedisplayshortskip=0pt
\belowdisplayshortskip=12pt

\subsection*{Краевая задача}

Рассмотрим дифференциальное уравнение второго порядка

\begin{equation}
    \ddot{x} + a(t)\dot{x} + b(t)x = f(t)
\end{equation}

Пусть функции $a(t)$, $b(t)$ и $f(t)$ определены
и непрерывны на некотором отрезке $t_0 \le t \le t_1$.
Будем искать решение $x(t)$ уравнения (1),
удовлетворяющее следующим условиям:

\begin{gather}
    \begin{split}
        \dot{x}(t_0) - \alpha_0 x(t_0) = \alpha_1, \\
        \dot{x}(t_1) - \beta_0 x(t_1) = \beta_1,
    \end{split}
\end{gather}

где $\alpha_0, \alpha_1, \beta_0, \beta_1$ - постоянные.

Условия (2) называются краевыми или граничными условиями,
а сама задача (1), (2) - краевой задачей.

Условия (2) не позволяют найти одновременно значение
$x(t)$ и $\dot{x}(t)$ ни при $t = t_0$, ни при $t = t_1$.
Поэтому краевая задача (1), (2) не сводится к задаче Коши.

Для краевой задачи (1), (2) может осуществляться
любая из трех возможностей:
она может иметь единственное решение, бесконечное
множество решений и вообще не иметь решений.

\begin{example*}
    Для дифференциального уравнения

    \begin{equation}
        \ddot{x} + x = 0
    \end{equation}

    общее решение которого имеет вид

    \begin{equation}
        x = c_1 \cos(t) + c_2 \sin(t)
    \end{equation}

    рассмотрим следующие три краевые задачи
    с граничными условиями:

    \begin{align}
        &\dot{x}(0) = 0, &\dot{x}(1) = 1; \\
        &\dot{x}(0) = 0, &\dot{x}(\pi) = 1; \\
        &\dot{x}(0) = 0, &\dot{x}(\pi) = 0;
    \end{align}

    Краевая задача (3), (5) имеет единственное решение,
    так как из формулы (4), в силу условий (5), получаем

    \begin{equation*}
        c_1 = - \frac{1}{\sin(t)}, \ \ \ \ \ \
        c_2 = 0,
    \end{equation*}

    Краевая задача (3), (6) не имеет решений,
    так как из формулы (4), в силу (6), следует что

    \begin{equation*}
        1 = -c_1 \sin(\pi) = 0
    \end{equation*}

    что невозможно.

    Краевая задача (3), (7) имеет бесконечное множество решений:

    \begin{equation*}
        x = c_1 \cos(t)
    \end{equation*}

    где $c_1$ - произвольная постоянная.

\end{example*}

\subsection*{Метод функции Грина}

Для уравнения

\begin{equation}
    \ddot{x} + a(t)\dot{x} + b(t)x = f(t)
\end{equation}

рассмотрим краевую задачу с однородными граничными условиями
%добавить замечание про неоднородные условия

\begin{gather}
\begin{split}
    & \alpha_0 \dot{x}(t_0) + \alpha_1 x(t_0) = 0
    \ \ \ \ (\alpha^2_0 + \alpha^2_1 > 0) \\
    & \beta_0 \dot{x}(t_1) + \beta_1 x(t_1) = 0
    \ \ \ \ (\beta^2_0 + \beta^2_1 > 0)
\end{split}
\end{gather}

\begin{remark*}
    Любое неоднородное граничное условие можно преобразовать
    к однородному. Пусть дана краевая задача с неоднородными
    граничными условиями

    \begin{align*}
    & \alpha_0 \dot{x}(t_0) + \alpha_1 x(t_0) = \alpha_2
        \ \ \ \ (\alpha^2_0 + \alpha^2_1 > 0) \\
    & \beta_0 \dot{x}(t_1) + \beta_1 x(t_1) = \beta_2
    \ \ \ \ (\beta^2_0 + \beta^2_1 > 0)
    \end{align*}

    Пусть $\omega(t)$ произвольная непрерывная функция
    удовлетворяющая неоднородным граничным условиям.
    Сделаем замену $x(t) = y(t) + \omega(t)$.
    Получим

    \begin{equation*}
        \ddot{y} + a(t)\dot{y} + b(t)y =
        f(t) - \ddot{\omega}(t) - a(t)\dot{\omega}(t) -
        b(t)\omega(t) = f_1(t)
    \end{equation*}

    \begin{gather*}
        \begin{aligned}
            & \alpha_0 \dot{y}(t_0) + \alpha_1 y(t_0) +
            \alpha_0 \dot{\omega}(t_0) + \alpha_1 \omega(t_0)
            = \alpha_2 \\
            & \beta_0 \dot{y}(t_1) + \beta_1 y(t_1) +
            \beta_0 \dot{\omega}(t_1) + \beta_1 \omega(t_1)
            = \beta_2
        \end{aligned}
        \iff
        \begin{aligned}
            & \alpha_0 \dot{y}(t_0) + \alpha_1 y(t_0) = 0 \\
            & \beta_0 \dot{y}(t_1) + \beta_1 y(t_1) = 0
        \end{aligned}
    \end{gather*}

    Поэтому далее будем предпологать что замена сделана.

\end{remark*}

\mbox{} \\

Будем предпологать, что рассматриваемая краевая задача
имеет единственное решение.

Пусть $x = x_1(t)$ - какое-либо нетривиальное решение
однородного уравнения

\begin{equation}
    \ddot{x} + a(t) \dot{x} + b(t)x = 0
\end{equation}

удовлетворяющее первому из граничных условий (9),

\begin{equation*}
    \alpha_0 \dot{x_1}(t_0) + \alpha_1 x_1(t_0) = 0
\end{equation*}

а $x = x_2(t)$ - нетривиальное решение уравнения (10),
удовлетворяющее второму граничному условию

\begin{equation*}
    \beta_0 \dot{x_2}(t_1) + \beta_1 x_2(t_1) = 0
\end{equation*}

Тогда $x_1(t)$ не удовлетворяет второму граничному условию,
так как в противном случае при любой постоянной $c$
функции $x = c x_1(t)$ были бы решениями краевой задачи (10),
(9), и наша исходная краевая задача (8), (9) имела бы
бесконечное множетво решений. Точно так же доказывается, что
$x_2(t)$ не удовлетворяет первому граничному условию. Итак,

\begin{gather}
    \begin{split}
    \beta_0 \dot{x_1}(t_1) + \beta_1 x_1(t_1) \ne 0 \\
    \alpha_0 \dot{x_2}(t_0) + \alpha_1 x_2(t_0) \ne 0
    \end{split}
\end{gather}

Постороенные решения $x = x_1(t)$, $x = x_2(t)$ линейно
независимы, так как в противном случае они были бы
пропорциональны и потому удовлетворяли бы одним и тем же
граничным условиям, что невозможно.

Решение неоднородного уравнения (8) будем искать
методом вариации постоянных. Записывая решение $x(t)$
в виде

\begin{equation}
    x(t) = c_1(t) x_1(t) + c_2(t) x_2(t)
\end{equation}

для определения функций $\dot{c}_1(t)$ и $\dot{c}_2(t)$
получим следующую систему линейных уравнений:

\begin{align*}
    & \dot{c}_1(t) x_1(t) + \dot{c}_2(t) x_2(t) = 0 \\
    & \dot{c}_1(t) \dot{x}_1(t) + \dot{c}_2(t) \dot{x}_2(t) = f(t)
\end{align*}

Решение этой систмы уравнений имеет вид

\begin{gather}
    \begin{split}
        & \dot{c}_1(t) = - \frac{x_2(t)f(t)}{w(t)}, \\
        & \dot{c}_2(t) = \frac{x_1(t)f(t)}{w(t)},
    \end{split}
\end{gather}

где

\begin{equation*}
    w(t) =
    \begin{vmatrix}
        x_1(t) & x_2(t) \\
        \dot{x}_1(t) & \dot{x}_2(t)
    \end{vmatrix}
    \ne 0
\end{equation*}

- определитель Вронского, составленный для линейно
независиых решений $x_1(t)$ и $x_2(t)$. Интегрируя
соотношения (13), получим

\begin{align*}
    & c_1(t) = - \int^t_{t_1} \frac{x_2(s)f(s)}{w(s)} ds + \gamma_1
    = \int^{t_1}_t \frac{x_2(s)f(s)}{w(s)} ds + \gamma_1, \\
    & c_2(t) = \int^t_{t_0} \frac{x_1(s)f(s)}{w(s)} ds + \gamma_2
\end{align*}

где $\gamma_1$ и $\gamma_2$ - постоянные.
Подставляя найденные выражения для $c_1(t)$ и $c_2(t)$ в (12),
получим общее решение уравнения (8)

\begin{equation}
    x(t) =
    x_1(t)\int^{t_1}_t \frac{x_2(s)f(s)}{w(s)} ds +
    x_2(t)\int^t_{t_0} \frac{x_1(s)f(s)}{w(s)} ds
    + \gamma_1 x_1(t)
    + \gamma_2 x_2(t)
\end{equation}

Дифференцируя (14) по $t$, имеем

\begin{equation}
    \dot{x}(t) =
    \dot{x}_1(t)\int^{t_1}_t \frac{x_2(s)f(s)}{w(s)} ds +
    \dot{x}_2(t)\int^t_{t_0} \frac{x_1(s)f(s)}{w(s)} ds
    + \gamma_1 \dot{x}_1(t)
    + \gamma_2 \dot{x}_2(t)
\end{equation}

Потребуем теперь, чтобы решение (14) удовлетворяло
граничным условиям (9). Подставляя выражения (14), (15)
в первое из граничных условий (9) получим
(так как $\alpha_0 \dot{x}_1(t_0) + \alpha_1 x_1(t_0) = 0$)

\begin{align*}
    & 0 =
    (\alpha_0 \dot{x}_1(t_0) + \alpha_1 x_1(t_0))
    \int^{t_1}_t \frac{x_2(s)f(s)}{w(s)} ds +
    \gamma_1
    (\alpha_0 \dot{x}_1(t_0) + \alpha_1 x_1(t_0)) + \\
    & + \gamma_2(\alpha_0 \dot{x}_2(t_0) + \alpha_1 x_2(t_0)) =
    \gamma_2(\alpha_0 \dot{x}_2(t_0) + \alpha_1 x_2(t_0))
\end{align*}

В силу  неравенств (11) это возможно только при $\gamma_2 = 0$.
Подобным же образом доказывается, что $\gamma_1 = 0$.
Итак, решение краевой задачи (8), (9) можно
представить в виде

\begin{equation*}
    x(t) =
    x_1(t)\int^{t_1}_t \frac{x_2(s)f(s)}{w(s)} ds +
    x_2(t)\int^t_{t_0} \frac{x_1(s)f(s)}{w(s)} ds
\end{equation*}

или

\begin{equation}
    x(t) = \int^{t_1}_{t_0} G(t, s) f(s) ds
\end{equation}

где

\begin{equation}
    G(t,s) =
    \begin{cases}
        \frac{x_1(s) x_2(t)}{w(s)} , &t_0 \le s \le t,\\
        \frac{x_1(t) x_2(s)}{w(s)} , &t \le s \le t_1,\\
        \frac{x_1(s) x_2(t)}{w(s)} , &t_0 \le t \le s,\\
        \frac{x_1(t) x_2(s)}{w(s)} , &s \le t \le t_1,\\
    \end{cases}
\end{equation}

Построенная функция $G(t,s)$ называется функцией Грина
краевой задачи (8), (9). Таким образом, если функция
Грина найдена, то решение краевой задачи (8), (9) задается
формулой (16). Сама функция Грина от $f(t)$ не зависит
(она определяется решениями $x_1(t)$ и $x_2(t)$ однородного
уравнения (10)).
Легко проверить, что функция Грина $G(t,s)$ при любом
фиксированном $s$ обладает следующими свойствами:

\begin{enumerate}
    \item при $t \ne s$ \ $G(t,s)$ удовлетворяет однородному
        уравнению (10);
    \item при $t = t_0$ и $t = t_1$ \ $G(t,s)$ удовлетворяет
        соответственно первому и второму граничным условиям (9);
    \item при $t = s$ \ $G(t,s)$ непрерывна;
    \item $t = s$ производная $G'_t(t,s)$ имеет скачок, равный 1:
        \begin{equation*}
            G'_t \vert_{t = s + 0} - G'_t \vert_{t = s - 0} = 1
        \end{equation*}
\end{enumerate}

Свойства 1. - 3. проверяются совсем просто. Докажем свойство 4.
В силу (17)

\begin{equation*}
    G'_t =
    \begin{cases}
        \frac{\dot{x}_1(t) x_2(s)}{w(s)}, &t_0 \le t \le s, \\
        \frac{x_1(s) \dot{x}_2(t)}{w(s)}, &s \le t \le t_1,
    \end{cases}
\end{equation*}

откуда следует, что

\begin{equation*}
    G'_t \vert_{t = s + 0} - G'_t \vert_{t = s - 0} =
    \frac{x_1(s) \dot{x}_2(s) - \dot{x}_1(s) x_2(s)}{w(s)} = 1.
\end{equation*}

\begin{example*}

    Построим функцию Грина для краевой задачи

    \begin{equation*}
        \ddot{x} - \dot{x} = f(t), \ \ \ x(0) = 0,
        \ \ \ \dot{x}(1) = 0,
    \end{equation*}

    Общее решение однородного уравнения
    $\ddot{x} - \dot{x} = 0$ имеет вид $x = c_1 + c_2 e^t$,
    откуда находим $x_1(t) = 1 - e^t$ (нетривиальное решение,
    удовлетворяющее первому граничному условию), $x_2(t) = 1$,
    (нетривиальное решение, удовлетворяющее второму граничному
    условию),

    \begin{equation*}
        w(t) =
        \begin{vmatrix}
            1 - e^t & 1 \\
            -e^t & 0
        \end{vmatrix} = e^t
    \end{equation*}

    В силу формулы (17)

    \begin{equation*}
        G(t,s) =
        \begin{cases}
            (1 - e^t)e^{-s}, &0 \le t \le s, \\
            e^{-s} - 1, &s \le t \le 1,
        \end{cases}
    \end{equation*}


\end{example*}


\end{document}
