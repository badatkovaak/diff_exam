\documentclass{article}
\usepackage[T2A, T1]{fontenc}
\usepackage{fancyhdr}
\usepackage{setspace}
\usepackage{titlesec}
\usepackage{amsmath}
\usepackage{amssymb}
\usepackage{amsthm}
\usepackage[english, russian]{babel}

\setlength{\parindent}{0pt}
\theoremstyle{plain} \newtheorem*{theorem*}{Теорема}
\theoremstyle{plain} \newtheorem{theorem}{Теорема}[section]
% \theoremstyle{remark} \newtheorem{definition}{Definition}[section]
% \theoremstyle{remark} \newtheorem{notation}{Notation}[section]
% \theoremstyle{definition} \newtheorem{corollary}{Corollary}[section]


\begin{document}

\begin{theorem*}
    Suppose $x = \varphi (t)$ of ds (1) $\varphi (t) $ hi $ - \infty < t < +\infty$
    \begin{enumerate}
        \item $\varphi (t) = a$, ae $\varphi (t)$
        \item $T > 0$, $\varphi (t + T) = \varphi (t)$ for all $t$, $0 < \vert t_1 - t_2 \vert < T$, $\varphi (t_1) \neq \varphi(t_2)$
    \end{enumerate}
    in case 2 \ $x = \varphi(t)$
\end{theorem*}

\begin{proof}

    $x = \varphi(t)$ is def $a < t < b$. $\exists t_1,t_2 \in (a,b) \ \ (t_1 > t_2)$ such that $\varphi(t_1) = \varphi(t_2)$

    \begin{equation}
        \varphi(t) = \varphi(t + c)
    \end{equation}

    where $c = t_1 - t_2 > 0$
    $x = \varphi(t+c)$ , $a-c < t < b-c$, $a < t < b-c$
    \begin{equation}
        x = \psi(t) =
        \begin{cases}
            \varphi(t), &a < t < b,\\
            \varphi(t + c), & a-c < t \le a,
        \end{cases}
    \end{equation}
    $x = \varphi(t)$, $(a-c, b)$, $x = \varphi(t)$,
    $(-\infty, +\infty)$,
    $\varphi(t) = \varphi(t - c)$,
\end{proof}

\end{document}
