\documentclass{article}
\usepackage[T2A, T1]{fontenc}
\usepackage{fancyhdr}
\usepackage{setspace}
\usepackage{titlesec}
\usepackage{amsmath}
\usepackage{amssymb}
\usepackage{amsthm}
\usepackage[english, russian]{babel}

\setlength{\parindent}{0pt}
\setlength{\parskip}{0pt}

\theoremstyle{definition} \newtheorem*{theorem*}{Теорема}
\theoremstyle{plain} \newtheorem{theorem}{Теорема}[section]
% \theoremstyle{remark} \newtheorem{definition}{Definition}[section]
% \theoremstyle{remark} \newtheorem{notation}{Notation}[section]
\theoremstyle{definition} \newtheorem*{corollary*}{Следствие}
\theoremstyle{definition} \newtheorem{corollary}{Следствие}[section]


\begin{document}

\abovedisplayskip=0pt
\belowdisplayskip=9pt
\abovedisplayshortskip=0pt
\belowdisplayshortskip=12pt

\subsection*{Динамические системы}

Рассмотрим нормальную систему дифференциальных уравнений

\begin{equation}
    \dot{\bar{x}} = \bar{f}(\bar{x})
\end{equation}

правая часть которой не зависит от переменного $t$

Системы дифференциальных уравнений вида (1) называются
\textit{динамическими} или \textit{автономными}.

Предположим, что функция $\bar{f}(\bar{x})$ непрерывна на
некотором открытом множестве $D$ пространства переменных
$x^1, \dots, x^n$ и удовлетворяет условию Липшица в любом
замкнутом ограниченном подмножестве $D$. Тогда
в силу теорем существования и единственности, для любого
действительного числа $t_0$ и для любой точки $\bar{x}_0 \in D$
будет существовать единственное решение

\begin{equation*}
    \bar{x} = \bar{\varphi}(t)
\end{equation*}

Системы уравнений (1), удовлетворяющее условию

\begin{equation*}
    \bar{\varphi}(t_0) = \bar{x}_0
\end{equation*}

В пространстве переменных $x^1, \dots, x^n$ любое
решение $\bar{x} = \bar{\varphi}(t)$ динамической системы (1)
определяет кривую. Эту кривую с заданным на ней параметром
$t$ будем называть \textit{траекторией}.
Само пространство $x^1, \dots, x^n$ называется
\textit{фазовым пространством}.

\subsection*{Свойства решений динамических систем}

1. Если $ \bar{x} = \bar{\varphi}(t)$ -
решение динамической системы

\begin{equation}
    \dot{\bar{x}} = \bar{f}(\bar{x})
\end{equation}

то, для любого $c$, $\bar{x} = \bar{\varphi}(t + c)$
также является решением.
\begin{proof}
Следует из равенств

\begin{equation*}
    \frac{d}{dt} \bar{\varphi}(t + c)
    = \bar{\varphi}(t + c)
    = \bar{f}(\bar{\varphi}(t + c))
\end{equation*}

\end{proof}

2. Если $\bar{x} = \bar{\varphi}(t)$ и $\bar{x} =
\bar{\psi}(t)$ - два решения системы (1) и
$\bar{\varphi}(t_1) = \bar{\psi}(t_2)$,
то $\bar{\psi}(t) = \bar{\varphi}(t + c)$,
где $c = t_1 - t_2$.
Иначе говоря, если траектории $\bar{x} = \bar{\varphi}(t)$ и
$\bar{x} = \bar{\psi}(t)$ имеют общую точку,
то эти траектории совпадают.

\begin{proof}
В силу свойства 1, $\bar{x} = \bar{\varphi}(t + c)$ \
($c = t_1 - t_2$) - решение системы (1),
а в силу равенства $\bar{\varphi}(t_1) = \bar{\psi}(t_2)$,

\begin{equation*}
    \bar{\varphi}(t_2 + c)
    = \bar{\varphi}(t_1)
    = \bar{\psi}(t_2)
\end{equation*}

Таким образом, решения $\bar{x} = \bar{\varphi}(t + c)$
и $\bar{x} = \bar{\psi}(t)$
удовлетворяют одинаковым начальным
условиям при $t = t_2$ и, в силу теоремы единственности,
совпадают, т.е.

\begin{equation*}
    \bar{\varphi}(t + c) = \bar{\psi}(t)
\end{equation*}

\end{proof}

3. Решения динамической системы обладают групповым свойством:
если $\bar{x} = \bar{\varphi}(t, \bar{x}_0)$ - решение
системы (1), удовлетворяющее начальному условию
$\bar{\varphi}(0, x_0) =\bar{x}_0$, то

\begin{equation*}
    \bar{\varphi}(t, \bar{\varphi}(s, \bar{x}_0))
    = \bar{\varphi}(t + s, \bar{x}_0)
\end{equation*}

\begin{proof}
Положим $\bar{x}_1 = \bar{\varphi}(s, \bar{x}_0)$.
Тогда $\bar{\varphi}_1(t) = \bar{\varphi}(t, \bar{x}_1)$
- решение системы (1) и, в силу свойства 1,
$\bar{\varphi}_2 = \bar{\varphi}(t + s, \bar{x}_0)$
также является решением (1); при этом

\begin{align*}
    &\bar{\varphi}_1(0)
    = \bar{\varphi}(0, \bar{x}_1)
    = \bar{x}_1,\\
    &\bar{\varphi}_2(0)
    = \bar{\varphi}(s, \bar{x}_0)
    = \bar{x}_1.
\end{align*}

Таким образом, решения $\bar{\varphi}_1(t)$ и
$\bar{\varphi}_2(t)$ системы уравнений (1) удовлетворяют
одинаковым начальным условиям. В силу теоремы единственности
$\bar{\varphi}(t) = \bar{\varphi}_2(t)$ или

\begin{equation*}
    \bar{\varphi}(t, \bar{\varphi}(s, \bar{x}_0))
    = \bar{\varphi}(t + s, \bar{x}_0)
\end{equation*}

\end{proof}

Решение системы (1) вида $\bar{x} = a$, где $a$
- постоянный вектор, называется
\textit{положением равновесия} или \textit{точкой покоя}.

Очевидно, что если $\bar{x} = a$ - положение равновесия,
то $f(a) = 0$, и наоборот,
если $f(a) = 0$, то $\bar{x} = a$ - положение равновесия.

\subsection*{Множество периодов решения Д.С.}

Пусть $\bar{x} = \bar{\varphi}(t)$ - решение динамической
системы (1), определенное при $- \infty < t < + \infty$.
Число $c$ называется \textit{периодом решения}
$\bar{x} = \bar{\varphi}(t)$, если $\bar{\varphi}(t + c) =
\bar{\varphi}(t)$ при всех $t$.

Обозначим $F$ множество всех периодов решения
$\bar{x} = \bar{\varphi}(t)$ (это множество непусто,
так как $0 \in F$). Докажем следующие свойства множества $F$.
\\

1. Если $c \in F$, то $-c \in F$.
\begin{proof}
Так как $c$ - период, то
$\bar{\varphi}(t + c) = \bar{\varphi}(t)$.
Заменяя в этом равенстве $t$ на $t - c$, получим
$\bar{\varphi}(t) = \bar{\varphi}(t - c)$,
т.е. $-c$ является периодом.
\end{proof}

2. Если $c_1 \in F$, $c_2 \in F$, то $c_1 + c_2 \in F$.

\begin{proof}
Следует из равенств

\begin{equation*}
    \bar{\varphi}(t + c_1 + c_2)
    = \bar{\varphi}(t + c_1)
    = \bar{\varphi}(t)
\end{equation*}

\end{proof}

3. $F$ - замкнутое множество.
\begin{proof}
Пусть $c_n$ - сходящаяся последовательность периодов
и $\lim_{n \rightarrow \infty} c_n = c_0$. Тогда в силу
непрерывности $ $ имеем

\begin{equation*}
    \bar{\varphi}(t + c_0)
    = \bar{\varphi}(t + \lim_{n \rightarrow \infty} c_n)
    = \lim_{n \rightarrow \infty} \bar{\varphi}(t + c_n)
    = \bar{\varphi}(t)
\end{equation*}

Таким образом $c_0 \in F$, и, следовательно,
$F$ - замкнутое множество.
\end{proof}

\subsection*{Виды траекторий}

\begin{theorem*}
    Пусть траектория $\bar{x} = \bar{\varphi} (t)$ динамической системы
    (1) сама себя пересекает.
    Тогда решение $\bar{\varphi} (t)$ может быть продолжено на интервал
    $- \infty < t < +\infty$ и имеет место одна
    из следующих возможностей:
    \begin{enumerate}
        \item $\bar{\varphi} (t) = a$, т.е. решение $\bar{\varphi} (t)$
            является положением равновесия;
        \item существует такое число $T > 0$, что
            $\bar{\varphi} (t + T) = \bar{\varphi} (t)$
            при всех $t$, но при $0 < \vert t_1 - t_2 \vert < T$,
            $\bar{\varphi} (t_1) \neq \bar{\varphi}(t_2)$
    \end{enumerate}
    в случае 2 решение \ $\bar{x} = \bar{\varphi}(t)$ называется
    \textit{периодическим}, а его траектория
    - \textit{замкнутой траекторией} или \textit{циклом}.
\end{theorem*}

\begin{proof}
    Пусть решение $\bar{x} = \bar{\varphi}(t)$ опреелено при $a < t < b$.
    По предположению траектория решение сама себя пересекает,
    т.е. сущствуют такие
    $t_1,t_2 \in (a,b) ,\ \ (t_1 > t_2)$ что
    $\bar{\varphi}(t_1) = \bar{\varphi}(t_2)$

    В силу свойства 2 решений динамических систем,

    \begin{equation}
        \bar{\varphi}(t) = \bar{\varphi}(t + c)
    \end{equation}

    где $c = t_1 - t_2 > 0$.
    Функция $\bar{x} = \bar{\varphi}(t+c)$ является решением системы (1),
    определенным при $a-c < t < b-c$, и, кроме того в силу (2),
    эти решения совпадают на общей части их областей определения,
    т.е. при $a < t < b-c$. Следовательно, решение

    \begin{equation}
        \bar{x} = \bar{\psi}(t) =
        \begin{cases}
            \bar{\varphi}(t), &a < t < b,\\
            \bar{\varphi}(t + c), & a-c < t \le a,
        \end{cases}
    \end{equation}

    является продолжением решения $\bar{x} = \bar{\varphi}(t)$
    на интервал $(a-c, b)$.
    Последовательно повторяя описанную процедуру,
    получим продолжение
    решения $\bar{x} = \bar{\varphi}(t)$, определенное
    на интервале $(- \infty, b)$.
\newline
    С помощью равенства $\bar{\varphi}(t)
    = \bar{\varphi}(t - c)$, которое
    получается из (2) заменой $t$ на $t - c$, получим продолжение
    решения $\bar{x} = \bar{\varphi}(t)$ с интервала
    $(- \infty, b)$ на всю числовую ось $(-\infty, +\infty)$.

    Итак решение $\bar{x} = \bar{\varphi}(t)$  можно считать
    определенным при $- \infty < t < + \infty$ , причем,
    как ясно из самого способа продолжения, постоянная
    $c = t_1 - t_2 > 0$ является периодом этого решения.

    Пусть $F$ - множество периодов решения
    $\bar{x} = \bar{\varphi}(t)$.
    Могут представится две возможности:

    \begin{itemize}
        \item[а)] $F$ содержит сколь угодно малые
            положительные числа,
        \item[б)] в $F$ найдется наименьшее положительное
            число $T$.
    \end{itemize}

    В случае а) существует сходящаяся к нулю
    последовательность положительных периодов $c_n$.
    Пусть $t$ - произвольное действительное число.
    Дробные части

    \begin{equation*}
        \alpha_n = \frac{t}{c_n} - \left[ \frac{t}{c_n} \right]
    \end{equation*}

    чисел $\frac{t}{c_n}$ образуют ограниченную
    последовательность, а так как $c_n \rightarrow 0$, то

    \begin{equation*}
        \lim_{n \rightarrow \infty}
        \left\{ t - \left[ \frac{t}{c_n}\right]\right\}
        = \lim_{n \rightarrow \infty} (\alpha_n c_n) = 0
    \end{equation*}

    Числа $\left[ \frac{t}{c_n} \right] c_n$,
    будучи целыми кратными периодов $c_n$,
    сами являются периодами решения $\bar{\varphi}(t)$. Поэтому

    \begin{equation*}
        \bar{\varphi}(t)
        = \bar{\varphi}
        \left( t - \left[\frac{t}{c_n} c_n \right]\right)
    \end{equation*}

    переходя в равенстве (3) к пределу при
    $n \rightarrow \infty$, получим

    \begin{equation*}
        \bar{\varphi}(t)
        = \lim_{n \rightarrow \infty} \bar{\varphi}
        \left( t - \left[\frac{t}{c_n}\right] c_n \right)
        = \bar{\varphi} \left( \lim_{n \rightarrow \infty}
        \left( t - \left[ \frac{t}{c_n} \right] c_n
        \right)\right)
        = \bar{\varphi}(0)
    \end{equation*}

    Таким образом, решение $\bar{x} = \bar{\varphi}(t)$
    в случае а) является положением равновесия.
\\

    В случае б)

    \begin{equation*}
        \bar{\varphi}(t + T) = \bar{\varphi}(t)
    \end{equation*}

    Покажем, что $\bar{\varphi}(t_1) \ne \bar{\varphi}(t_2)$
    при $0 < \vert t_1 - t_2 \vert < T$.
    Предположим противное. Тогда найдутся
    такие $t_1$, $t_2$ ($0 < \vert t_1 - t_2 \vert < T$),
    что $\bar{\varphi}(t_1) = \bar{\varphi}(t_2)$. В силу свойства 2,
    $ \bar{\varphi}(t) = \bar{\varphi}(t + c)$,
    где $c = t_1 - t_2 \ne 0$.
    Таким образом, $c = t_1 - t_2$ служит периодом
    решения $\bar{\varphi}(t)$. В силу свойства 1
    множества $F$, положительное число
    $\vert t_1 - t_2 \vert = \pm c$ также является
    периодом, а это противоречит
    предположению, что $T$ - наименьший положительный
    период решения $\bar{\varphi}(t)$.
\end{proof}

    Из доказанной теоремы непосредственно получаем следующее

    \begin{corollary*}
    Траектория любого непродолжаемого
    решения динамической системы (1) может быть либо
    положением равновесия, либо замкнутой
    траекторией, либо траекторией без самопересечений.
    \end{corollary*}

\end{document}
