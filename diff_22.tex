\documentclass{article}
\usepackage[T2A, T1]{fontenc}
\usepackage{fancyhdr}
\usepackage{setspace}
\usepackage{titlesec}
\usepackage{amsmath}
\usepackage{amssymb}
\usepackage{amsthm}
\usepackage[english, russian]{babel}

\setlength{\parindent}{0pt}
\setlength{\parskip}{0pt}

\theoremstyle{plain} \newtheorem*{theorem*}{Теорема}
\theoremstyle{plain} \newtheorem{theorem}{Теорема}[section]
% \theoremstyle{remark} \newtheorem{definition}{Definition}[section]
% \theoremstyle{remark} \newtheorem{notation}{Notation}[section]
% \theoremstyle{definition} \newtheorem{corollary}{Corollary}[section]


\begin{document}

\abovedisplayskip=0pt
\belowdisplayskip=9pt
\abovedisplayshortskip=0pt
\belowdisplayshortskip=9pt

1. Если = - решение динамической системы

====

то, для любого =, = также является решением.
Доказательство. следует из равенств

2. Если = и = - два решения системы (1) и =, то =, где =.
Иначе говоря, если траектории = и = имеют общую точку,
то эти траектории совпадают.
Доказательство. В силу свойства 1, = (=) - решение системы (1),
а в силу равенства =,

====

Таким образом, решения = и = удовлетворяют одинаковым начальным
условиям при = и, в силу теоремы единственности, совпадают, т.е.

====

3. Решения динамической системы обладают групповым свойством:
если = - решение системы (1), удовлетворяющее начальному условию =, то

====

Доказательство. Положим =. Тогда = - решение системы (1) и,
в силу свойства 1, = также является решением (1); при этом


====

Таким образом, решения = и = системы уравнений (1) удовлетворяют
одинаковым начальным условиям. В силу теоремы единственности = или

====

Решение системы (1) вида =, где = - постоянный вектор, называется
положением равновесия или точкой покоя.

Очевидно, что если = - положение равновесия, то =, и наоборот,
если =, то = - положение равновесия.

Пусть = - решение динамической системы (1), определенное при =.
Число = называется перешения =, если = при всех =.

Обозначим = множество всех периодов решения =
(это множество непусто, так как =).
Докажем следующие свойства множества =.

1. Если =, то =.
Д. Так как = - период, то =. Заменяя в этом равенстве = на =, получим
=, т.е. = является периодом.

2. Если =, =, то =.
Д. следует из равенств

====

3. = - замкнутое множество.
Д. Пусть = - сходящаяся последовательность периодов и =. Тогда в силу
непрерывности = имеем

====

Таким образом =, и, следовательно, = - замкнутое множество.


\begin{theorem*}
    Пусть траектория $x = \varphi (t)$ динамической системы (1) сама себя пересекает.
    Тогда решение $\varphi (t)$ может быть продолжено на интервал
    $- \infty < t < +\infty$ и имеет место одна из следующих возможностей:
    \begin{enumerate}
        \item $\varphi (t) = a$, т.е. решение $\varphi (t)$
            является положением равновесия;
        \item существует такое число $T > 0$, что $\varphi (t + T) = \varphi (t)$
            при всех $t$, но при $0 < \vert t_1 - t_2 \vert < T$,
            $\varphi (t_1) \neq \varphi(t_2)$
    \end{enumerate}
    в случае 2 решение \ $x = \varphi(t)$ называется периодическим,
    а его траектория - замкнутой траекторией или циклом.
\end{theorem*}

\begin{proof}

    Пусть решение $x = \varphi(t)$ опреелено при $a < t < b$.
    По предположению траектория решение сама себя пересекает, т.е. сущствуют такие
    $t_1,t_2 \in (a,b) ,\ \ (t_1 > t_2)$ что $\varphi(t_1) = \varphi(t_2)$

    В силу свойства 2 решений динамических систем,

    \begin{equation}
        \varphi(t) = \varphi(t + c)
    \end{equation}

    где $c = t_1 - t_2 > 0$.
    Функция $x = \varphi(t+c)$ является решением системы (1),
    определенным при $a-c < t < b-c$, и, кроме того в силу (2),
    эти решения совпадают на общей части их областей определения,
    т.е. при $a < t < b-c$. Следовательно, решение

    \begin{equation}
        x = \psi(t) =
        \begin{cases}
            \varphi(t), &a < t < b,\\
            \varphi(t + c), & a-c < t \le a,
        \end{cases}
    \end{equation}

    является продолжением решения $x = \varphi(t)$ на интервал $(a-c, b)$.
    Последовательно повторяя описанную процедуру, получим продолжение
    решения $x = \varphi(t)$, определенное на интервале $(- \infty, b)$.
\newline
    С помощью равенства $\varphi(t) = \varphi(t - c)$, которое
    получается из (2) заменой $t$ на $t - c$, получим продолжение
    решения $x = \varphi(t)$ с интервала $(- \infty, b)$ на всю
    числовую ось $(-\infty, +\infty)$.

    Итак решение =  можно считать определенным при = , причем,
    как ясно из самого способа продолжения, постоянная = является
    периодом этого решения.

    Пусть = - множество периодов решения =. Могут представится две возможности:
    а - = содержит сколь угодно малые положительные числа,
    б - в = найдется наименьшее положительное число =.
    В случае а) существует сходящаяся к нулю последовательность положительных
    периодов =.
    Пусть = - произвольное действительное число. Дробные части

    ====

    чисел = образуют ограниченную последовательность, а так как =, то

    ====

    Числа = , будучи целыми кратными периодов =, свми являются периодами
    решения =. Поэтому

    ====

    переходя в равенстве (3) к пределу при =, получим

    ====

    Таким образом, решение = в случае а) является положением равновесия.

    В случае б)

    ====

    Покажем, что = при =. Предположим противное. Тогда найдутся такие
    =, = (=), что =. В силу свойства 2, =, где =. Таким образом,
    = служит периодом решения =. В силу свойства 1 множества =,
    положительное число = также является периодом, а это противоречит
    предположению, что = - наименьший положительный период решения =.
    Теорема Доказана
    Из доказанной теоремы непосредственно получаем следующее

    Следствие. Траектория любого непродолжаемого решения динамической
    системы (1) может быть либо положением равновесия, либо замкнутой
    траекторией, либо траекторией без самопересечений.



\end{proof}

\end{document}
