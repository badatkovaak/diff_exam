\documentclass{article}
\usepackage[T2A, T1]{fontenc}
\usepackage{fancyhdr}
\usepackage{setspace}
\usepackage{titlesec}
\usepackage{amsmath}
\usepackage{amssymb}
\usepackage{amsthm}
\usepackage[english, russian]{babel}

\setlength{\parindent}{0pt}
\setlength{\parskip}{0pt}

\theoremstyle{plain} \newtheorem*{theorem*}{Теорема}
\theoremstyle{plain} \newtheorem{theorem}{Теорема}[section]
% \theoremstyle{remark} \newtheorem{definition}{Definition}[section]
% \theoremstyle{remark} \newtheorem{notation}{Notation}[section]
\theoremstyle{definition} \newtheorem*{corollary*}{Следствие}
\theoremstyle{definition} \newtheorem{corollary}{Следствие}[section]


\begin{document}

\abovedisplayskip=0pt
\belowdisplayskip=9pt
\abovedisplayshortskip=0pt
\belowdisplayshortskip=12pt

1. Если $ x = \varphi(t)$ - решение динамической системы

\begin{equation}
    \dot{x} = f(x)
\end{equation}

то, для любого $c$, $x = \varphi(t + c)$
также является решением.
\begin{proof}
Следует из равенств

\begin{equation*}
    \frac{d}{dt} \varphi(t + c) = \varphi(t + c) =
    f(\varphi(t + c))
\end{equation*}

\end{proof}

2. Если $x = \varphi(t)$ и $x = \psi(t)$ -
два решения системы (1) и $\varphi(t_1) = \psi(t_2)$,
то $\psi(t) = \varphi(t + c)$, где $c = t_1 - t_2$.
Иначе говоря, если траектории $x = \varphi(t)$ и
$x = \psi(t)$ имеют общую точку,
то эти траектории совпадают.

\begin{proof}
В силу свойства 1, $x = \varphi(t + c)$
($c = t_1 - t_2$) - решение системы (1),
а в силу равенства $\varphi(t_1) = \psi(t_2)$,

\begin{equation*}
    \varphi(t_2 + c) = \varphi(t_1) = \psi(t_2)
\end{equation*}

Таким образом, решения $x = \varphi(t + c)$ и $x = \psi(t)$
удовлетворяют одинаковым начальным
условиям при $t = t_2$ и, в силу теоремы единственности,
совпадают, т.е.

\begin{equation*}
    \varphi(t + c) = \psi(t)
\end{equation*}

\end{proof}

3. Решения динамической системы обладают групповым свойством:
если $x = \varphi(t, x_0)$ - решение системы (1),
удовлетворяющее начальному условию $\varphi(0, x_0) = x_0$, то

\begin{equation*}
    \varphi(t, \varphi(s, x_0)) = \varphi(t + s, x_0)
\end{equation*}

\begin{proof}
Положим $x_1 = \varphi(s, x_0)$.
Тогда $\varphi_1(t) = \varphi(t, x_1)$ - решение системы (1) и,
в силу свойства 1, $\varphi_2 = \varphi(t + s, x_0)$
также является решением (1); при этом

\begin{align*}
    &\varphi_1(0) = \varphi(0, x_1) = x_1,\\
    &\varphi_2(0) = \varphi(s, x_0) = x_1.
\end{align*}

Таким образом, решения $\varphi_1(t)$ и $\varphi_2(t)$
системы уравнений (1) удовлетворяют
одинаковым начальным условиям. В силу теоремы единственности
$\varphi(t) = \varphi_2(t)$ или

\begin{equation*}
    \varphi(t, \varphi(s, x_0)) = \varphi(t + s, x_0)
\end{equation*}

\end{proof}

Решение системы (1) вида $x = a$, где $a$
- постоянный вектор, называется
положением равновесия или точкой покоя.

Очевидно, что если $x = a$ - положение равновесия,
то $f(a) = 0$, и наоборот,
если $f(a) = 0$, то $x = a$ - положение равновесия.

Пусть $x = \varphi(t)$ - решение динамической системы (1),
определенное при $- \infty < t < + \infty$.
Число $c$ называется периодом решения $x = \varphi(t)$,
если $\varphi(t + c) = \varphi(t)$ при всех $t$.

Обозначим $F$ множество всех периодов решения $x = \varphi(t)$
(это множество непусто, так как $0 \in F$).
Докажем следующие свойства множества $F$.
\\

1. Если $c \in F$, то $-c \in F$.
\begin{proof}
Так как $c$ - период, то $\varphi(t + c) = \varphi(t)$.
Заменяя в этом равенстве $t$ на $t - c$, получим
$\varphi(t) = \varphi(t - c)$, т.е. $-c$ является периодом.
\end{proof}

2. Если $c_1 \in F$, $c_2 \in F$, то $c_1 + c_2 \in F$.

\begin{proof}
Следует из равенств

\begin{equation*}
    \varphi(t + c_1 + c_2) = \varphi(t + c_1) = \varphi(t)
\end{equation*}

\end{proof}

3. $F$ - замкнутое множество.
Д. Пусть $c_n$ - сходящаяся последовательность периодов
и $\lim_{n \rightarrow \infty} c_n = c_0$. Тогда в силу
непрерывности $ $ имеем

\begin{equation*}
    \varphi(t + c_0) =
    \varphi(t + \lim_{n \rightarrow \infty} c_n) =
    \lim_{n \rightarrow \infty} \varphi(t + c_n) =
    \varphi(t)
\end{equation*}

Таким образом $c_0 \in F$, и, следовательно,
$F$ - замкнутое множество.


\begin{theorem*}
    Пусть траектория $x = \varphi (t)$ динамической системы
    (1) сама себя пересекает.
    Тогда решение $\varphi (t)$ может быть продолжено на интервал
    $- \infty < t < +\infty$ и имеет место одна
    из следующих возможностей:
    \begin{enumerate}
        \item $\varphi (t) = a$, т.е. решение $\varphi (t)$
            является положением равновесия;
        \item существует такое число $T > 0$, что
            $\varphi (t + T) = \varphi (t)$
            при всех $t$, но при $0 < \vert t_1 - t_2 \vert < T$,
            $\varphi (t_1) \neq \varphi(t_2)$
    \end{enumerate}
    в случае 2 решение \ $x = \varphi(t)$ называется
    периодическим, а его траектория
    - замкнутой траекторией или циклом.
\end{theorem*}

\begin{proof}

    Пусть решение $x = \varphi(t)$ опреелено при $a < t < b$.
    По предположению траектория решение сама себя пересекает,
    т.е. сущствуют такие
    $t_1,t_2 \in (a,b) ,\ \ (t_1 > t_2)$ что
    $\varphi(t_1) = \varphi(t_2)$

    В силу свойства 2 решений динамических систем,

    \begin{equation}
        \varphi(t) = \varphi(t + c)
    \end{equation}

    где $c = t_1 - t_2 > 0$.
    Функция $x = \varphi(t+c)$ является решением системы (1),
    определенным при $a-c < t < b-c$, и, кроме того в силу (2),
    эти решения совпадают на общей части их областей определения,
    т.е. при $a < t < b-c$. Следовательно, решение

    \begin{equation}
        x = \psi(t) =
        \begin{cases}
            \varphi(t), &a < t < b,\\
            \varphi(t + c), & a-c < t \le a,
        \end{cases}
    \end{equation}

    является продолжением решения $x = \varphi(t)$
    на интервал $(a-c, b)$.
    Последовательно повторяя описанную процедуру,
    получим продолжение
    решения $x = \varphi(t)$, определенное на интервале
    $(- \infty, b)$.
\newline
    С помощью равенства $\varphi(t) = \varphi(t - c)$, которое
    получается из (2) заменой $t$ на $t - c$, получим продолжение
    решения $x = \varphi(t)$ с интервала $(- \infty, b)$ на всю
    числовую ось $(-\infty, +\infty)$.

    Итак решение $x = \varphi(t)$  можно считать
    определенным при $- \infty < t < + \infty$ , причем,
    как ясно из самого способа продолжения, постоянная
    $c = t_1 - t_2 > 0$ является периодом этого решения.

    Пусть $F$ - множество периодов решения $x = \varphi(t)$.
    Могут представится две возможности:

    \begin{itemize}
        \item[а)] $F$ содержит сколь угодно малые положительные числа,
        \item[б)] в $F$ найдется наименьшее положительное число $T$.
    \end{itemize}

    В случае а) существует сходящаяся к нулю
    последовательность положительных периодов $c_n$.
    Пусть $t$ - произвольное действительное число. Дробные части

    \begin{equation*}
        \alpha_n = \frac{t}{c_n} - \left[ \frac{t}{c_n} \right]
    \end{equation*}

    чисел $\frac{t}{c_n}$ образуют ограниченную
    последовательность, а так как $c_n \rightarrow 0$, то

    \begin{equation*}
        \lim_{n \rightarrow \infty}
        \left\{ t - \left[ \frac{t}{c_n}\right]\right\} =
        \lim_{n \rightarrow \infty} (\alpha_n c_n) = 0
    \end{equation*}

    Числа $\left[ \frac{t}{c_n} \right] c_n$,
    будучи целыми кратными периодов $c_n$,
    сами являются периодами решения $\varphi(t)$. Поэтому

    \begin{equation*}
        \varphi(t) =
        \varphi \left( t - \left[\frac{t}{c_n} c_n \right]\right)
    \end{equation*}

    переходя в равенстве (3) к пределу при
    $n \rightarrow \infty$, получим

    \begin{equation*}
        \varphi(t) =
        \lim_{n \rightarrow \infty} \varphi
        \left( t - \left[\frac{t}{c_n}\right] c_n \right) =
        \varphi \left( \lim_{n \rightarrow \infty}
        \left( t - \left[ \frac{t}{c_n} \right] c_n
        \right)\right) =
        \varphi(0)
    \end{equation*}

    Таким образом, решение $x = \varphi(t)$ в случае а)
    является положением равновесия.
\\

    В случае б)

    \begin{equation*}
        \varphi(t + T) = \varphi(t)
    \end{equation*}

    Покажем, что $\varphi(t_1) \ne \varphi(t_2)$
    при $0 < \vert t_1 - t_2 \vert < T$.
    Предположим противное. Тогда найдутся
    такие $t_1$, $t_2$ ($0 < \vert t_1 - t_2 \vert < T$),
    что $\varphi(t_1) = \varphi(t_2)$. В силу свойства 2,
    $ \varphi(t) = \varphi(t + c)$,
    где $c = t_1 - t_2 \ne 0$.
    Таким образом, $c = t_1 - t_2$ служит периодом
    решения $\varphi(t)$. В силу свойства 1
    множества $F$, положительное число
    $\vert t_1 - t_2 \vert = \pm c$ также является
    периодом, а это противоречит
    предположению, что $T$ - наименьший положительный
    период решения $\varphi(t)$.
\end{proof}

    Из доказанной теоремы непосредственно получаем следующее

    \begin{corollary*}
    Траектория любого непродолжаемого
    решения динамической системы (1) может быть либо
    положением равновесия, либо замкнутой
    траекторией, либо траекторией без самопересечений.
    \end{corollary*}

\end{document}
